
\documentclass[12pt]{article}
%%%%%%%%%%%%%%%%%%%%%%%%%%%%%%%%%%%%%%%%%%%%%%%%%%%%%%%%%%%%%%%%%%%%%%%%%%%%%%%%%%%%%%%%%%%%%%%%%%%%%%%%%%%%%%%%%%%%%%%%%%%%%%%%%%%%%%%%%%%%%%%%%%%%%%%%%%%%%%%%%%%%%%%%%%%%%%%%%%%%%%%%%%%%%%%%%%%%%%%%%%%%%%%%%%%%%%%%%%%%%%%%%%%%%%%%%%%%%%%%%%%%%%%%%%%%
\usepackage{amsfonts}
\usepackage{amssymb}
\usepackage{sw20elba}

%TCIDATA{OutputFilter=LATEX.DLL}
%TCIDATA{Version=5.50.0.2890}
%TCIDATA{<META NAME="SaveForMode" CONTENT="1">}
%TCIDATA{BibliographyScheme=Manual}
%TCIDATA{Created=Monday, July 01, 2013 14:26:16}
%TCIDATA{LastRevised=Thursday, August 08, 2013 12:28:26}
%TCIDATA{<META NAME="GraphicsSave" CONTENT="32">}
%TCIDATA{<META NAME="DocumentShell" CONTENT="Articles\SW\mrvl">}
%TCIDATA{CSTFile=LaTeX article (bright).cst}

\newtheorem{theorem}{Theorem}
\newtheorem{axiom}[theorem]{Axiom}
\newtheorem{claim}[theorem]{Claim}
\newtheorem{conjecture}[theorem]{Conjecture}
\newtheorem{corollary}[theorem]{Corollary}
\newtheorem{definition}[theorem]{Definition}
\newtheorem{example}[theorem]{Example}
\newtheorem{exercise}[theorem]{Exercise}
\newtheorem{lemma}[theorem]{Lemma}
\newtheorem{notation}[theorem]{Notation}
\newtheorem{problem}[theorem]{Problem}
\newtheorem{proposition}[theorem]{Proposition}
\newtheorem{remark}[theorem]{Remark}
\newtheorem{solution}[theorem]{Solution}
\newtheorem{summary}[theorem]{Summary}
\newenvironment{proof}[1][Proof]{\noindent\textbf{#1.} }{{\hfill $\Box$ \\}}
\input{tcilatex}
\addtolength{\textheight}{30pt}

\begin{document}

\title{Descendants of algebra 5.1 of order $p^{7}$}
\author{Michael Vaughan-Lee}
\date{July 2013}
\maketitle

The following occurs in computing the immediate descendants of order $p^{7}$
of algebra 5.1. There are 6 commutator structures possible with $L^{2}$
having order $p^{2}$, and this problem arises in Case 6, with $pL=L^{2}$.
Here $pa=pd=0$, and we write%
\[
\left( 
\begin{array}{l}
pb \\ 
pc \\ 
pe%
\end{array}%
\right) =A\left( 
\begin{array}{l}
ba \\ 
ca%
\end{array}%
\right) 
\]%
for a $3\times 2$ matrix $A$. We consider the orbits of matrices 
\[
A=\left( 
\begin{array}{ll}
u & v \\ 
t & x \\ 
y & z%
\end{array}%
\right) 
\]%
where $(tz-xy)^{2}-(ux-vt)(uz-vy)$ is not a square under the action of
non-singular matrices $\left( 
\begin{array}{ll}
a & c \\ 
b & d%
\end{array}%
\right) $ given by 
\[
\left( 
\begin{array}{ll}
u & v \\ 
t & x \\ 
y & z%
\end{array}%
\right) \rightarrow (ad-bc)^{-2}\left( 
\begin{array}{lll}
(ad+bc) & 2bd & -2ac \\ 
cd & d^{2} & -c^{2} \\ 
-ab & -b^{2} & a^{2}%
\end{array}%
\right) \left( 
\begin{array}{ll}
u & v \\ 
t & x \\ 
y & z%
\end{array}%
\right) \left( 
\begin{array}{ll}
d & -b \\ 
-c & a%
\end{array}%
\right) .
\]

Each such orbit contains a matrix with $u=0$ and $v=1$, and we pick one
matrix of this form out of each orbit, giving $k$ algebras 
\[
\langle
a,b,c,d,e\,|%
\,da,ea,cb,db-ca,eb,dc,ec,ed-ba,pa,pb-ca,pc-tba-xca,pd,pe-yba-zca,\,\text{%
class }2\rangle ,
\]%
where $k=4$ when $p=3$, $k=(p^{2}-1)/2$ when $p=1\func{mod}3$, and $%
k=(p^{2}+1)/2$ when $p=2\func{mod}3$.

First we consider the action of four particular matrices $\left( 
\begin{array}{ll}
a & c \\ 
b & d%
\end{array}%
\right) $: $\left( 
\begin{array}{ll}
1 & 0 \\ 
b & 1%
\end{array}%
\right) $, $\left( 
\begin{array}{ll}
1 & c \\ 
0 & 1%
\end{array}%
\right) $, $\left( 
\begin{array}{ll}
a & 0 \\ 
0 & d%
\end{array}%
\right) $, $\left( 
\begin{array}{ll}
0 & c \\ 
b & 0%
\end{array}%
\right) $. These four matrices transform $\left( 
\begin{array}{ll}
u & v \\ 
t & x \\ 
y & z%
\end{array}%
\right) $ into

\begin{equation}
\allowbreak \left( 
\begin{array}{cc}
u+2tb & v+2xb-b\left( u+2tb\right)  \\ 
t & x-tb \\ 
-tb^{2}-ub+y & z-vb-xb^{2}+b\left( tb^{2}+ub-y\right) 
\end{array}%
\right) ,
\end{equation}

\begin{equation}
\allowbreak \left( 
\begin{array}{cc}
u-2yc-c\left( v-2zc\right)  & v-2zc \\ 
t+uc-c\left( -zc^{2}+vc+x\right) -yc^{2} & -zc^{2}+vc+x \\ 
y-zc & z%
\end{array}%
\right) ,
\end{equation}

\begin{equation}
\allowbreak \left( 
\begin{array}{cc}
\frac{u}{a} & \frac{v}{d} \\ 
\frac{t}{a^{2}}d & \frac{x}{a} \\ 
\frac{y}{d} & z\frac{a}{d^{2}}%
\end{array}%
\right) ,
\end{equation}

\begin{equation}
\allowbreak \left( 
\begin{array}{cc}
-\frac{v}{b} & -\frac{u}{c} \\ 
\frac{z}{b^{2}}c & \frac{y}{b} \\ 
\frac{x}{c} & t\frac{b}{c^{2}}%
\end{array}%
\right) .
\end{equation}

\bigskip From (1) we see that we can take $u=0$ provided $t\neq 0$, and from
(2) and (4) we see that we can take $v=0$ provided $z\neq 0$, and then swap $%
u$ and $v$ to get $u=0$. In the case when $t=z=0$ and both $u$ and $v$ are
non-zero we can use (4) to take $x=y=1$. (None of the rows of $\left( 
\begin{array}{ll}
u & v \\ 
t & x \\ 
y & z%
\end{array}%
\right) $ can equal zero.)

Now consider the action of $\left( 
\begin{array}{ll}
a & c \\ 
-a & c%
\end{array}%
\right) $ on $\left( 
\begin{array}{ll}
u & v \\ 
0 & 1 \\ 
1 & 0%
\end{array}%
\right) $. We obtain%
\[
\allowbreak \frac{1}{4a^{2}c^{2}}\left( 
\begin{array}{cc}
0 & -4a^{2}c \\ 
-c\left( c^{2}-uc^{2}\right) -c\left( c^{2}+vc^{2}\right)  & a\left(
c^{2}+vc^{2}\right) -a\left( c^{2}-uc^{2}\right)  \\ 
c\left( a^{2}+ua^{2}\right) +c\left( a^{2}-va^{2}\right)  & a\left(
a^{2}+ua^{2}\right) -a\left( a^{2}-va^{2}\right) 
\end{array}%
\right) .
\]

This proves that every orbit contains a matrix with first row $(0,1)$.

Now in a matrix $\left( 
\begin{array}{ll}
0 & 1 \\ 
t & x \\ 
y & z%
\end{array}%
\right) $, the condition \textquotedblleft $(tz-xy)^{2}-(ux-vt)(uz-vy)$ is
not a square\textquotedblright\ reduces to \textquotedblleft $(tz-xy)^{2}-ty$
is not a square\textquotedblright , so neither $t$ nor $y$ can be zero. The
action of $\left( 
\begin{array}{ll}
a & 0 \\ 
0 & 1%
\end{array}%
\right) $ on $\left( 
\begin{array}{ll}
0 & 1 \\ 
t & x \\ 
y & z%
\end{array}%
\right) $ gives $\allowbreak \left( 
\begin{array}{cc}
0 & 1 \\ 
\frac{t}{a^{2}} & \frac{x}{a} \\ 
y & za%
\end{array}%
\right) $, and so every orbit contains a matrix $\left( 
\begin{array}{ll}
0 & 1 \\ 
t & x \\ 
y & z%
\end{array}%
\right) $ where $t$ is either one or the least non-square modulo $p$, where $%
0\leq x\leq \frac{p-1}{2}$, and where when $x=0$, $0\leq z\leq \frac{p-1}{2}$%
. It seems experimentally that every orbit contains a matrix with $u=0$, $%
v=t=1$, but I have no proof of this.$\allowbreak $

Next we show that if we have $(u,v,t,x,y,z)$ satisfying these conditions,
and if we act on $\left( 
\begin{array}{ll}
u & v \\ 
t & x \\ 
y & z%
\end{array}%
\right) $ with a non-identity matrix $\left( 
\begin{array}{ll}
a & 0 \\ 
b & d%
\end{array}%
\right) $, then we obtain $\left( 
\begin{array}{ll}
u^{\prime } & v^{\prime } \\ 
t^{\prime } & x^{\prime } \\ 
y^{\prime } & z^{\prime }%
\end{array}%
\right) $ where $(u^{\prime },v^{\prime },t^{\prime },x^{\prime },y^{\prime
},z^{\prime })$ which is lexicographically higher than $(u,v,t,x,y,z)$. The
action of $\left( 
\begin{array}{ll}
a & 0 \\ 
b & d%
\end{array}%
\right) $ on $\left( 
\begin{array}{ll}
0 & 1 \\ 
t & x \\ 
y & z%
\end{array}%
\right) $ gives

\[
\allowbreak \left( 
\begin{array}{cc}
2\frac{t}{a^{2}}b & a\left( \frac{1}{ad}+2\frac{x}{a^{2}}\frac{b}{d}\right)
-2\frac{t}{a^{2}}\frac{b^{2}}{d} \\ 
\frac{t}{a^{2}}d & \frac{x}{a}-\frac{t}{a^{2}}b \\ 
d\left( \frac{y}{d^{2}}-\frac{t}{a^{2}}\frac{b^{2}}{d^{2}}\right)  & 
-a\left( \frac{1}{a}\frac{b}{d^{2}}-\frac{z}{d^{2}}+\frac{x}{a^{2}}\frac{%
b^{2}}{d^{2}}\right) -b\left( \frac{y}{d^{2}}-\frac{t}{a^{2}}\frac{b^{2}}{%
d^{2}}\right) 
\end{array}%
\right) ,
\]%
which is lexicographically higher unless $b=0$ and $d=1$. But when $b=0$ and 
$d=1$, then the action gives $\allowbreak \left( 
\begin{array}{cc}
0 & 1 \\ 
\frac{t}{a^{2}} & \frac{x}{a} \\ 
y & za%
\end{array}%
\right) $, which is lexicographically higher unless $a=1$.

So we only need to consider the action of matrices $\left( 
\begin{array}{ll}
a & c \\ 
b & d%
\end{array}%
\right) $ where $c\neq 0$, and we write such a matrix as $k\left( 
\begin{array}{ll}
a & 1 \\ 
b & d%
\end{array}%
\right) $. The action of $\left( 
\begin{array}{ll}
a & 1 \\ 
b & d%
\end{array}%
\right) $ on $\left( 
\begin{array}{ll}
0 & 1 \\ 
t & x \\ 
y & z%
\end{array}%
\right) $ gives

\[
\allowbreak \frac{1}{\left( b-ad\right) ^{2}}\left( 
\begin{array}{cc}
a(2z-2dy-d)+b(2td^{2}-1-2xd) & b\left( 2ya-2tbd\right) +a\left(
b-2za+ad+2xbd\right)  \\ 
z-d-xd^{2}-d\left( y-td^{2}\right)  & a\left( xd^{2}+d-z\right) +b\left(
y-td^{2}\right)  \\ 
ab+xb^{2}-za^{2}-d\left( tb^{2}-ya^{2}\right)  & b\left(
tb^{2}-ya^{2}\right) -a\left( -za^{2}+ab+xb^{2}\right) 
\end{array}%
\right) .
\]%
$\allowbreak \allowbreak $

So we need $a(2z-2dy-d)+b(2td^{2}-1-2xd)=0$ and we want to take%
\[
k=\allowbreak \frac{1}{\left( b-ad\right) ^{2}}\left( b\left(
2ya-2tbd\right) +a\left( b-2za+ad+2xbd\right) \right) .
\]

The \textsc{magma} program note2dec5.1.m finds a set of representatives for
the orbits. The integer parameters $t,x,y,z$ correspond to $t1,x1,y1,z1$ in
GF$(p)$.

\end{document}
