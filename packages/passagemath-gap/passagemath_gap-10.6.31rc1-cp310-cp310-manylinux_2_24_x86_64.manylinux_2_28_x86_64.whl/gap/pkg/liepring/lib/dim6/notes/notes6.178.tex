
\documentclass[12pt]{article}
%%%%%%%%%%%%%%%%%%%%%%%%%%%%%%%%%%%%%%%%%%%%%%%%%%%%%%%%%%%%%%%%%%%%%%%%%%%%%%%%%%%%%%%%%%%%%%%%%%%%%%%%%%%%%%%%%%%%%%%%%%%%%%%%%%%%%%%%%%%%%%%%%%%%%%%%%%%%%%%%%%%%%%%%%%%%%%%%%%%%%%%%%%%%%%%%%%%%%%%%%%%%%%%%%%%%%%%%%%%%%%%%%%%%%%%%%%%%%%%%%%%%%%%%%%%%
\usepackage{amsfonts}
\usepackage{amssymb}
\usepackage{sw20elba}

%TCIDATA{OutputFilter=LATEX.DLL}
%TCIDATA{Version=5.50.0.2890}
%TCIDATA{<META NAME="SaveForMode" CONTENT="1">}
%TCIDATA{BibliographyScheme=Manual}
%TCIDATA{Created=Thursday, June 27, 2013 17:23:01}
%TCIDATA{LastRevised=Friday, June 28, 2013 11:25:03}
%TCIDATA{<META NAME="GraphicsSave" CONTENT="32">}
%TCIDATA{<META NAME="DocumentShell" CONTENT="Articles\SW\mrvl">}
%TCIDATA{CSTFile=LaTeX article (bright).cst}
%TCIDATA{ComputeDefs=
%$A=\left( 
%\begin{array}{cc}
%t & x \\ 
%y & z%
%\end{array}%
%\right) $
%$P=\left( 
%\begin{array}{ll}
%-1 & 0 \\ 
%0 & 1%
%\end{array}%
%\right) $
%}


\newtheorem{theorem}{Theorem}
\newtheorem{axiom}[theorem]{Axiom}
\newtheorem{claim}[theorem]{Claim}
\newtheorem{conjecture}[theorem]{Conjecture}
\newtheorem{corollary}[theorem]{Corollary}
\newtheorem{definition}[theorem]{Definition}
\newtheorem{example}[theorem]{Example}
\newtheorem{exercise}[theorem]{Exercise}
\newtheorem{lemma}[theorem]{Lemma}
\newtheorem{notation}[theorem]{Notation}
\newtheorem{problem}[theorem]{Problem}
\newtheorem{proposition}[theorem]{Proposition}
\newtheorem{remark}[theorem]{Remark}
\newtheorem{solution}[theorem]{Solution}
\newtheorem{summary}[theorem]{Summary}
\newenvironment{proof}[1][Proof]{\noindent\textbf{#1.} }{{\hfill $\Box$ \\}}
\input{tcilatex}
\addtolength{\textheight}{30pt}

\begin{document}

\title{Algebra 6.178}
\author{Michael Vaughan-Lee}
\date{June 2013}
\maketitle

Algebra 6.178 has four paramaters $x,y,z,t$ taking all integer values,
subject to $A=\left( 
\begin{array}{cc}
t & x \\ 
y & z%
\end{array}%
\right) $ being non-singular modulo $p$. Two such parameter matrices $A$ and 
$B$ define isomorphic algebras if and only if%
\[
B=\frac{1}{\det P}PAP^{-1}\func{mod}p
\]%
for some matrix $P$ of the form%
\begin{equation}
\left( 
\begin{array}{ll}
\alpha  & \beta  \\ 
\omega \beta  & \alpha 
\end{array}%
\right) \text{ or }\left( 
\begin{array}{ll}
\alpha  & \beta  \\ 
-\omega \beta  & -\alpha 
\end{array}%
\right) 
\end{equation}%
which is non-singular modulo $p$. (Here, as elsewhere, $\omega $ is a
primitive element modulo $p$.) So we need to compute the orbits of GL$(2,p)$
under the action of the subroup of GL$(2,p)$ consisting of matrices of the
form (1). The set of all matrices $P$ of this form is a group $G$ of order $%
2(p^{2}-1)$. The number of orbits is $p^{2}+(p+1)/2-\gcd (p-1,4)/2$.

We show that every orbit contains a matrix $\left( 
\begin{array}{cc}
0 & x \\ 
y & z%
\end{array}%
\right) $ or $\left( 
\begin{array}{cc}
1 & x \\ 
y & z%
\end{array}%
\right) $.

Let $A=\left( 
\begin{array}{cc}
t & x \\ 
y & z%
\end{array}%
\right) $.

If $P=\left( 
\begin{array}{ll}
\alpha  & 0 \\ 
0 & \alpha 
\end{array}%
\right) $ then $\frac{1}{\det P}PAP^{-1}=\allowbreak \left( 
\begin{array}{cc}
\frac{t}{\alpha ^{2}} & \frac{x}{\alpha ^{2}} \\ 
\frac{y}{\alpha ^{2}} & \frac{z}{\alpha ^{2}}%
\end{array}%
\right) $. This implies that we can take $t=0$ or $1$ provided $t$ is a
square.

If $P=\left( 
\begin{array}{ll}
\alpha & 0 \\ 
0 & -\alpha%
\end{array}%
\right) $ then $\frac{1}{\det P}PAP^{-1}=\allowbreak \left( 
\begin{array}{cc}
-\frac{t}{\alpha ^{2}} & \frac{x}{\alpha ^{2}} \\ 
\frac{y}{\alpha ^{2}} & -\frac{z}{\alpha ^{2}}%
\end{array}%
\right) $, which means that you can take $t=0$ or 1 unless $-1$ is a square,
i.e. unless $p=1\func{mod}4$.

If $P=\left( 
\begin{array}{ll}
0 & \beta  \\ 
\omega \beta  & 0%
\end{array}%
\right) $ then $\frac{1}{\det P}PAP^{-1}=\allowbreak \left( 
\begin{array}{cc}
-\frac{z}{\beta ^{2}\omega } & -\frac{y}{\beta ^{2}\omega ^{2}} \\ 
-\frac{x}{\beta ^{2}} & -\frac{t}{\beta ^{2}\omega }%
\end{array}%
\right) $, so in the case $p=1\func{mod}4$ you can take $t=0$ or $1$
provided $t$ is a square or $z$ is not a square.

More generally, if $P=\left( 
\begin{array}{ll}
\alpha  & \beta  \\ 
\omega \beta  & \alpha 
\end{array}%
\right) $ then 
\begin{eqnarray*}
&&\frac{1}{\det P}PAP^{-1} \\
&=&\frac{1}{\left( \alpha ^{2}-\beta ^{2}\omega \right) ^{2}}\left( 
\begin{array}{cc}
t\alpha ^{2}+y\alpha \beta -x\alpha \beta \omega -z\beta ^{2}\omega  & 
x\alpha ^{2}-y\beta ^{2}-t\alpha \beta +z\alpha \beta  \\ 
y\alpha ^{2}-x\beta ^{2}\omega ^{2}+t\alpha \beta \omega -z\alpha \beta
\omega  & z\alpha ^{2}-y\alpha \beta -t\beta ^{2}\omega +x\alpha \beta
\omega 
\end{array}%
\right) .
\end{eqnarray*}%
$\allowbreak $So to show that we can take $t=0$ or $1$ even in the case $p=1%
\func{mod}4$, we need to show that whatever the values of $t,x,y,z$ we can
always find $\alpha ,\beta $ (not both zero) such that

\[
t\alpha ^{2}+y\alpha \beta -x\alpha \beta \omega -z\beta ^{2}\omega 
\]%
is a square. Clearly this is possible if $t$ is a square, or if $z$ is not a
square. So let $p=1\func{mod}4$, and assume that $t$ is not a square and
that $z$ is a square. We show that we can always find some value of $\alpha $
for which%
\[
t\alpha ^{2}+y\alpha -x\alpha \omega -z\omega 
\]%
is a square. (Since $z$ is a square, this value of $\alpha $ cannot be
zero.) Completing the square, we have%
\[
t\alpha ^{2}+y\alpha -x\alpha \omega -z\omega =t(\alpha +\frac{y-x\omega }{2t%
})^{2}-\frac{(y-x\omega )^{2}}{4t}-z\omega .
\]%
Setting $\frac{(y-x\omega )^{2}}{4t}+z\omega $ equal to $\lambda $, we see
that finding $\alpha $ such that $t\alpha ^{2}+y\alpha -x\alpha \omega
-z\omega $ is a square is equivalent to finding $\alpha $ such that 
\[
t\alpha ^{2}-\lambda 
\]%
is a square. If $\lambda $ is a square then (since $p=1\func{mod}4$) we see
that $t\alpha ^{2}-\lambda $ is a square when $\alpha =0$. On the other hand
if $\lambda $ is not a square then (since $t$ is not a square) we can find $%
\alpha $ such that $t\alpha ^{2}-\lambda =0$.

So we can assume that $t=0$ or 1, This means that we can find
representatives for the $p^{2}+(p+1)/2-\gcd (p-1,4)/2$ orbits in work of
order $p^{5}$. Not brilliant --- it would be nice to do better.

\end{document}
