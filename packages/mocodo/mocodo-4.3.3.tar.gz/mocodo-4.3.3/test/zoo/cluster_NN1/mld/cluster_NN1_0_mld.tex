\documentclass[a4paper]{article}
\usepackage[normalem]{ulem}
\usepackage[T1]{fontenc}
\usepackage[french]{babel}
\frenchsetup{StandardLayout=true}

\newcommand{\relat}[1]{\textsc{#1}}
\newcommand{\attr}[1]{#1}
\newcommand{\prim}[1]{\uline{#1}}
\newcommand{\foreign}[1]{\#\textsl{#1}}

\title{Conversion en relationnel\\du MCD \emph{cluster\_NN1}}
\author{\emph{Généré par Mocodo}}

\begin{document}
\maketitle

\begin{itemize}
  \item \relat{Réservation} (\prim{num résa}, \attr{arrhes}, \attr{date résa}, \foreign{num voilier}$^{u\_1}$, \foreign{num semaine}$^{u\_1}$, \attr{tarif})
  \begin{itemize}
    \item Le champ \emph{num résa} constitue la clé primaire de la table. C'était déjà un identifiant de l'entité \emph{Réservation}.
    \item Les champs \emph{arrhes} et \emph{date résa} étaient déjà de simples attributs de l'entité \emph{Réservation}.
    \item Le champ \emph{num voilier} est une clé étrangère. Il a migré par l'association de dépendance fonctionnelle \emph{Offrir} à partir de l'entité \emph{Voilier} en perdant son caractère identifiant. Il obéit en outre à la contrainte d'unicité 1.
    \item Le champ \emph{num semaine} est une clé étrangère. Il a migré par l'association de dépendance fonctionnelle \emph{Offrir} à partir de l'entité \emph{Semaine} en perdant son caractère identifiant. Il obéit en outre à la contrainte d'unicité 1.
    \item Le champ \emph{tarif} a migré à partir de l'association de dépendance fonctionnelle \emph{Offrir}.
  \end{itemize}

  \item \relat{Semaine} (\prim{num semaine}, \attr{date début}$^{u\_1}$)
  \begin{itemize}
    \item Le champ \emph{num semaine} constitue la clé primaire de la table. C'était déjà un identifiant de l'entité \emph{Semaine}.
    \item Le champ \emph{date début} était déjà un simple attribut de l'entité \emph{Semaine}. Il obéit à la contrainte d'unicité 1.
  \end{itemize}

  \item \relat{Voilier} (\prim{num voilier}, \attr{longueur})
  \begin{itemize}
    \item Le champ \emph{num voilier} constitue la clé primaire de la table. C'était déjà un identifiant de l'entité \emph{Voilier}.
    \item Le champ \emph{longueur} était déjà un simple attribut de l'entité \emph{Voilier}.
  \end{itemize}

\end{itemize}

\end{document}
